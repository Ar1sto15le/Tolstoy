
\chapter{}
\centerline{Providence in the "Life of Humanity": A Critical Analysis of Tolstoy's Calculus}

% \pagestyle{plain}

\label{}

\section{What is the problem of Suffering?}
Providence in the “Life of Humanity”: A Critical Analysis of Tolstoy’s Calculus
1. Introduction and Statement of Thesis
Tolstoy's War and Peace is both a novel and an argument on the proper interpretation and methodology of history. This is particularly interesting because these forms — the novel and the argument — are at odds with each other in this sense: the novel is able to capture certain elements of life and daily existence that a philosophical argument is not able to capture. The novel, for example, is able to evoke strong emotions and empathy with the characters.  In this dichotomy of forms, the tension between independence of personality, also known as free will, and the laws of history is palpable.  The assertion of this paper is that this very dichotomy points to the difficulty of reducing human life, and consequently, human history to reason and the laws of nature.  In other words, Tolstoy's argument of the second epilogue, that "in the present case it is similarly necessary to renounce a freedom that does not exist, and to recognize a dependence of which we are not conscious", is not fully supported by the events of the novel itself.
 Instead, what emerges from this tension between the novel and the philosophical argument of historical interpretation is a nuanced account of providence which is characterized by these aspects: human beings have free will, but they can also fall into situations in which they lose this free will to a deterministic, machinistic force (such as the force that overcomes soldiers as they execute Platon Karataev); contrary to the arguments of the second epilogue, those who are in the proper state of mind are able to grow in inner freedom at precisely the time in which their external freedom is decreasing (Pierre as a prisoner and Andrew near his death); this state of mind, that allows for this inner freedom, is dependent on simple faith in God’s providence — faith that all is being shaped for a greater good despite not knowing how it is being shaped. 
2. Layout of the Paper
The layout of this paper is first to analyze the tension of the primary literary forms in the novel: the novel narrative and the historical-philosophical argument on the correct interpretation of history. From this analysis, it should become clear that the novel itself is not entirely resolved on giving up free will entirely in order to find historical laws. The novel itself affirms the presence of both historical, deterministic laws and the operation of free will. Thus, in order to resolve these two, the final conclusion is that there must be an operation of divine providence. Providence is the logical answer to this (false) dichotomy between free will and historical determinism as it allows for human free will, while at the same time allowing for history moving towards a particular purpose. As mentioned in the introductory paragraph, this divine providence has three particular characteristics that become clear in the novel. 
3. Dichotomy of Forms: the Inherent Tension Within War and Peace  
The second epilogue explicitly explains that the conception of free will that institutions and churches depend on —  a conception that is implicit in conceptions of the soul and good and evil — must necessarily be cast aside in order for the laws of inevitability to be found. "Just so it now seems as if we have only to admit the law of inevitability, to destroy the conception of the soul, of good and evil, and all the institutions of state and church that have been built on up on those conceptions." Yet, especially coming in the second epilogue, this necessity of casting aside free will seems to be quite at odds with novel that we have just read. In this novel, while there are certainly moments of inevitability, there are also certainly great moments of inner freedom and free will. In particular, there are areas in which free will manifests itself in beautiful and free human interactions. 
If the laws of inevitability are true, how are we to explain the transformative experiences, the good and wholesome marriages found in the first epilogue and the unexplainable love of General Bolkonski for his daughter?  Are these not presented in the novel as certain transcendent experiences of tenderness, joy and beauty?  These transcendent experiences are all the more transcendent because they are manifestations of free will. Pierre did not have to marry Natasha, for example. This is a marriage that he freely entered into, unlike his previous marriage with Helene. These are the very transcendent experiences, the "life of nations and of humanity", that Tolstoy admits he cannot fully capture with words. He himself admits that to "seize and put into words, to describe directly the life of humanity or even of a single nation, appears impossible."  And yet he still attempts to describe the ‘life of humanity’ in his novel. We are left, then, in a situation in which the very dichotomy between free will and historical determinism seems to be pulling the author in two ways.  This is represented in the dichotomy of forms: novel and philosophical argument.  This tension is what we will explore.  
4. A New Historical Method: the Differential of History
Perhaps the best place to begin is Tolstoy's argument for a certain historical method through which the continuous "movement of humanity, arising as it does from unnumerable arbitrary human wills" is to be understood. In the beginning of Book XI, Tolstoy cites an example of Achilles chasing a tortoise. In this paradox, Achilles is not able to catch the tortoise because he must always reach the point where the tortoise had been. But, by the time he has reached this point, the tortoise has already moved.  Tolstoy uses this as an example of the limitations of discrete mathematical reasoning.  This paradox is only solved once we have employed the study of continuous motion, which is also known as Newton's calculus.  
When we employ continuous thinking we realize that the error in the Achilles-Tortoise paradox lies in the fact that Achilles’ motion is faster than the Tortoise’s. In other words, Achilles velocity function is not defined as “catch up to the rabbit and stop”, as if it were operating in discrete and non-continuous chunks of time. His motion is actually defined as a continuous function over time. This being the case, there will be a point at which Achilles is at the same exact spot as the tortoise at the same exact time. In a position versus time graph, with position on the y-axis and time on the x-axis, we would see that the position-time curve of Achilles would eventually intersect with the position-time curve of the tortoise. After this point, because Achilles velocity is always greater than the tortoise’s velocity, his position will be ahead of the tortoise’s position. This is an intuitive conclusion that is resolved by a continuous understanding of physics and mathematics.
Similarly, Tolstoy argues, historical analysis must adopt a continuous method itself.  "Only by taking infinitesimally small units for observation (the differential of history, that is, the individual tendencies of men) and attaining to the art of integrating them (that is, finding the sum of these infinitesimals) can we hope to arrive at the laws of history." It is precisely this differential of history, the everyday actions of men, that is explored throughout the novel.  Consequently, it follows that the novel itself is also the integral of these infinitesimals. One law of history that Tolstoy claims to have found through this method of history is the randomness and chaos of military victory, as well as the proportional relationship between the spirit of the army and the size of the army.  "The spirit of an army is the factor which multiplied by the mass gives the resulting force."  This is, unsurprisingly, a historical formulation of the first law of Newtonian physics, Force = mass x acceleration, which coincided with Newton’s development of calculus.  It is this very spirit of the army that Kutuzov leverages in order to achieve military victory.
Despite these historical laws that arise from Tolstoy's historical calculus, it is the differential of calculus that seems to transcend the laws of inevitability.  Using Kutuzov as an example, we learn that by "long years of military experience he knew, and with the wisdom of age understood, that it is impossible for one man to direct hundreds of thousands of others struggling with death, and he knew that the result of a battle is decided not by the orders of a commander in chief ... but by that intangible force called the spirit of the army, and he watched this force and guided it in as far as that was in his power."  Notice the simplicity and humility in Kutuzov's acknowledgment of a force greater than himself, while at the same time the acknowledgement of the narrator that Kutuzov is actually affecting the battle by leveraging the spirit of the army. This implies that although there is a historical law at work, there is also an element of human free will at work at the same time.
This ability to operate with and leverage these deterministic laws of history lies in a unique characteristic of Kutuzov’s: his simplicity. Kutuzov does not trick himself into believing that he is a great man — great in the eyes of the historians, as Napoleon is.  Instead, Kutuzov is a very real model of humility. It is this very humility that enables him to direct and guide that force. It is also this humility that enables Kutuzov to be in touch with the spirit of the army. 
5. Free Will, Simplicity and Absurdity: an Introduction to Providence
There is a similar humility found in Mary, "how can we, miserable sinners that we are, know the terrible and holy secrets of Providence".  Mary is our first example of an explicit belief in Providence, as opposed to a belief in laws of inevitability. Her view of Providence is also opposed to the belief of the great military strategists who believe that they can control their own outcomes. This position of Providence is one that Pierre eventually comes to believe, as well. This is seen in his quote: "To endure war is the most difficult subordination of man's freedom to the law of God... Simplicity is submission to the will of God; you cannot escape from Him."  It is in fact this very virtue of simplicity that is raised in the everyday interactions, in the differential of history. As seen in Pierre’s words, simplicity is connected to belief in Providence. It is fitting, then, that Christ is the key of this simplicity, as the narrator explicitly tells us: "For us with the standard of good and evil given us by Christ, no human actions are incommensurable.  And there is no greatness where simplicity, goodness, and truth are absent."  This is the very same Christ who is being threatened by the laws of inevitability in history.
If these laws of inevitability do indeed threaten to "destroy the conception of the soul, of good and evil", how are we to resolve this dilemma?  An answer comes in a close analysis of the analogy to astronomy.  "... so also in history the new view says: "It is true that we are not conscious of our dependence, but by admitting our free will we arrive at absurdity, while by admitting our dependence on the external world, on time, and on cause, we arrive at laws."  The trouble here is that the historian, like a physicist, is seeking certainty and discounting the possibility of absurdity that is the manifestation of free will intersecting with providence. 
In other words, the narrator is arguing that the historian must ignore absurdity as simply something that we cannot yet account for with the laws of inevitability: “Free will is for history only an expression for the unknown remainder of what we know about the laws of human life.”  But to ignore the absurdity of the ‘life of humanity’ is to misinterpret the results of the very methodology of looking at the differential of history. Is absurdity and free will not what a close analysis of the daily existence, of the differential of history, has shown us?  Is it not absurd that Andrew's greatest moments of clarity and happiness come as he lay dying on the battlefield and in his death bed?  Is it not absurd that Pierre is finally able to find peace as a prisoner of war?  
6. Evaluating the Differentials of History Reveals Providence in the Life of Prince Andrew
If we are to take this novel as an example of the historical method of analyzing and summing the differentials of history, then we must also consider the description of the events that take place as though they are presented in order to be analyzed as the differentials of history. This being the case, it is especially important to witness the very 'absurdity' that is such a common occurrence throughout the novel.  The most powerful example of this 'absurdity' comes from Prince Andrew on his deathbed: "Yes, a new happiness was revealed to me of which man cannot be deprived..." Notice that this happiness was revealed and that this happiness is very important to man himself.  "A happiness lying beyond material forces, outside the material influences that act on man—a happiness of the soul alone, the happiness of loving."  Notice, too, that this happiness lies outside the realm of the material forces, the very forces on which historical determinism depends.
This happiness is made manifest through Christ: "Every man can understand it, but to conceive it and enjoin it was possible only for God… but how did God enjoin that law?  And why was the Son…?" It appears that Andrew is saying that all men can conceive of a happiness that is the result of true love — but “not love which loves for something, for some quality, for some purpose, or for some reason, but the love which I [Andrew] — while dying — first experienced when I saw my enemy and yet loved him.”   The sort of love that Andrew is talking about is the sort of love that Christ must have had for the Romans and the Jews as they persecuted him and hung him on a cross. Furthermore, the sort of experience that Andrew is talking about is one that men can conceive of, but cannot achieve without the aid of divine grace. This is shown in his next statement: “It is possible to love someone dear to you with human love, but an enemy can only be loved by divine love.”  In other words, Andrew has participated in Christ’s divine love and therein lies the source of his happiness which lies beyond material forces.
The sort of experience that Andrew is talking about can be further characterized by a sense of freely participating in providence. It is when the will has come to know how sinful it is and how grateful it should be that it is prepared to partake in the providential experience. This is why Andrew is prepared to partake in his last encounter with Natasha. This encounter is difficult to ascribe merely to chance. “He now understood for the first time all the cruelty of his rejection of her, the cruelty of his rupture with her.” It is of course at this very moment that Natasha, “whom of all people he most longed to love with this new pure divine love that had been revealed to him”, comes to his side.  His soul and his will have been prepared by his fatal wound on the battlefield to the point that he finally understands and experiences divine love. Having experienced this divine love before, Andrew freely longs to participate in and share this love with Natasha. This encounter reveals providence in both his battlefield experience as well as the fact that Natasha is present at his deathbed.
In Andrew's observations, then, we see in the differential of history the very absurdity of free will that we must necessarily dismiss in order to arrive at the laws of inevitability.  The novel as a manifestation of the historical calculus refutes the necessity of these laws.  Perhaps, then, a dichotomy is the wrong way of looking at the relationship between free will and inevitability — for only "by uniting them [free will and inevitability] do we get a clear conception of man's life." In what way might we unite free will and inevitability?  The logical conclusion is the synthesis of these two concepts which is known as providence — a way in which the world and its operations are guided by God without limiting the free will of humans. This theme of providence is explored, too, in the novel and is, in fact, the resolution to the tension of the forms (philosophy and novel). We saw the providence in Andrew’s death and his coming to divine law. This is one illustration of providence at work in the novel, in the closer analysis of the differential of history. Another great example of this providence at work is in Platon Karataev and his effect on Pierre.
7. Karataev and Imprisoned Pierre: an Explicit and Coherent Account of Providence
One of the first ways that we are introduced to Platon is in this description from Pierre who describes how Platon conceived of his own life: “his life, as he regarded it, had no meaning as a separate thing. It had meaning only as part of a whole of which he was always conscious… He could not understand the value or significance of any word or deed taken separately.” Every single action in his life and every word that he says are only understood in the context of the whole, which for Platon manifests itself as God’s providence. In Platon we have precisely the union of free will and inevitability.  Carrying on the description of Platon only understanding his words in the context of the whole, we see that there is certainly value in analyzing the differentials of historical calculus and the sum individually, but we should not lose sight of the fact that the differentials of history must be understood in the context of the whole. It is in this vein that Platon is able to see his punishment of being forced into the army as a blessing. He is able to see what a blessing it is that he was enlisted in the stead of his brother who was a family man, because he is aware of the greater purpose that is being served. Platon also illustrates very clearly how one is able to achieve this clarity: by recognizing that the ultimate destination is the afterlife and that God is guiding the historical particulars with this end in mind. It is with this perspective that Platon is able to find joy amidst his imprisonment and sickness and this perspective is best illustrated in his merchant story. 
 “God, it seems, has chastened me,” the innocent merchant says as he begins to describe his story.  He continues, upon hearing the actual murderer confess to committing the crime, “God will forgive you, we are all sinners in His sight. I  suffer for my own sins.”  Importantly, Karataev ends by saying that God forgave the man, in the form of of his death and consequently his reward in the afterlife. This story illustrates important characteristics of the synthesis between free will and providence that are corroborated both at Prince Andrew’s deathbed and in the lives of Platon and Pierre. 
The first characteristic is the sense that God, in His ultimate wisdom, is ‘chastening’ or purifying the souls of those who are suffering for the sake of the afterlife. This is the purification that Andrew undergoes on the battlefield. The second characteristic is that this purification or encounter with the divine prepares a soul in such a way that is ready to participate in providence: Andrew has been prepared to freely participate in divine, selfless love and the merchant is able to accept his lot in prison because he realizes that “we are all sinners in His sight”. Finally, note that Karataev’s view of providence implies a general faith that God is molding all things for the purpose of salvation, and in this way he is conscious of the end towards which all is aimed, but his knowledge does not extend to the particulars. It is a knowledge based in his faith of God. Karataev and his merchant may not know exactly in what way their imprisonment is going to help in their salvation, but they trust, generally, that God is guiding them through the imprisonment for the sake of their salvation. The final key is that this is a truth that is known not through the intellect, but through one’s “whole being” as it is learned in the selfless-divine love that Andrew experiences.
	Through his encounter with Karataev, Pierre is able to grasp, for the first time, the peace that he has been seeking. Ironically, despite Pierre’s best efforts to find this peace through society and intellectualism, through his military passions and in his desire to assassinate Napoleon as a great historical figure, Pierre finds this peace in the harshness that accompanies being a prisoner of war. This is a truth that he had learned “not with his intellect but with his whole being”: “that nothing in this world is terrible. He had learned that as there is no condition in which man can be happy and entirely free, so there is no condition in which he need be unhappy and lack freedom. He learned that suffering and freedom have their limits and that those limits are very near together.” 
Unpacking this ‘consolatory truth’, it seems that the key takeaway for Pierre is that freedom and happiness are connected, but in such a way that one is not excluded from the other — “there is no condition in which he need be unhappy and lack freedom” — in other words, there is not a situation in which it is necessary to be both unhappy and to lack freedom. This is the truth that Pierre experiences when he lacks freedom entirely as a prisoner while at the same time, and in fact because of his lack of external freedom, he finds the most happiness / internal freedom. And, yet, while this surrender in the physical sense is unwilled — Pierre does not will that he is taken prisoner — a physical lack of freedom can never capture his soul. “They hold me captive. What, me?  Me?  My immortal soul?”  
Furthermore, the mechanistic determinism is a force that Pierre can withstand and at the very moment when his external freedom is being taken away, his inner sense of independence grows greatest: “It was terrible [the callous force that causes men to commit murder against their wills], but he felt that in proportion to the efforts of that fatal force to crush him, there grew and strengthened in his soul a power of life independent of it.”  It is thus that suffering and freedom are so intertwined: at the limits of suffering, man is able come to know true freedom. This is the preparation that makes the soul ready to willingly embrace providence. Laws of determinism can never touch internal freedom. 
8. Summary and Conclusions
Perhaps the most comprehensive account of free will and providence, in light of suffering and, in fact, because of suffering, is Pierre’s rapturous revelation that: “Life is everything. Life is God. Everything changes and moves and that movement is God. And while there is life there is joy in consciousness of the divine. To love life is to love God. Harder and more blessed than all else is to love this life in one’s sufferings, in innocent sufferings.”  This statement shows the fundamental propositions of this providential view of life: life is God, meaning that God is guiding everything that moves towards His end, the revealed end of salvation; joy comes in consciousness of the divine purpose; that to ‘love life is to love God’, and here, in the concept of love, free will is manifested in its fullest sense. Love by its very definition cannot be coerced, it must be freely given. Suffering, particularly in the form of the loss of external freedom, through the Karataev lens might be thought a blessed trial. In this trial, the soul is prepared to participate more fully in God’s divine providence and love for Him grows stronger. It is this preparation of the soul, manifested in the loss external freedom for Andrew and Pierre, that brings them to realize providence in their own lives and to bring them into contact with divine, self-giving love. This self-giving love is most fully evident in Christ and his death, as Andrew and the narrator both explicitly state. This self-gift of Christ reveals the truth of Pierre’s realization that there is no situation in which external freedom must exclude inner freedom: in Christ’s death, in his complete loss of external freedom, the greatest internal freedom is attained for all of humanity.











%  vkp
